\documentclass[12pt]{article}

\usepackage[portuguese]{babel}
\usepackage[utf8]{inputenc}
\usepackage{mathtools}


\begin{document}
	\begin{titlepage}
		\centering
			{\scshape\LARGE Lista de exercícios de Introdução a Lógica \par}
		\vspace{4cm}
			{\Large\itshape Lorhan Sohaky\par}
		\vspace{0.3cm}
			{\scshape\itshape RA : 740951\par}
		\vfill
	{\large \today\par}
	\end{titlepage}
	
	\begin{enumerate}
	  \item \begin{enumerate}
	    		\item I[p]=V e I[q]=V; I[p]=F e I[q]=V; I[p]=F e I[q]=F
	    		\item I[p]=V e I[q]=V; I[p]=F e I[q]=F
	    		\item I[p]=V; I[p]=F
	    		\item I[p]=V e I[q]=F
	    		\item Impossível dar falso, já que sempre um será verdadeiro.
	    		\item I[p]=V e I[q]=F; I[p]=F e I[q]=V
	    		\item I[p]=V e I[q]=F; I[p]=V e I[q]=V; I[p]=F e I[q]=F;
	    		\item Impossível dar falso.
	    		\item I[\alpha]=?
	    		  
	  		\end{enumerate}
	\end{enumerate}

\end{document}
